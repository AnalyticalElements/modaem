%% LyX 1.1 created this file.  For more info, see http://www.lyx.org/.
%% Do not edit unless you really know what you are doing.
\documentclass[11pt,oneside,english]{book}
\usepackage{pslatex}
\usepackage[T1]{fontenc}
\usepackage[latin1]{inputenc}
\pagestyle{headings}
\usepackage{babel}
\setcounter{secnumdepth}{3}
\setcounter{tocdepth}{3}
\usepackage{setspace}
\onehalfspacing

\makeatletter


%%%%%%%%%%%%%%%%%%%%%%%%%%%%%% LyX specific LaTeX commands.
\providecommand{\LyX}{L\kern-.1667em\lower.25em\hbox{Y}\kern-.125emX\@}

%%%%%%%%%%%%%%%%%%%%%%%%%%%%%% Textclass specific LaTeX commands.
 \newenvironment{lyxcode}
   {\begin{list}{}{
     \setlength{\rightmargin}{\leftmargin}
     \raggedright
     \setlength{\itemsep}{0pt}
     \setlength{\parsep}{0pt}
     \verbatim@font}%
    \item[]}
   {\end{list}}

\makeatother

\begin{document}


\title{ModAEM Users Guide\\
Version 1.4pre2}


\author{Phil DiLavore\\
Vic Kelson\\
WHPA Inc.}


\date{August 10, 2001}

\maketitle
\tableofcontents{}

\listoffigures{}

\listoftables{}


\chapter{Introduction\label{chapter::introduction}}


\section{Supported Analytic Elements}

This section will describe the analytic elements supported by ModAEM.


\chapter{Package Description - Analytic Elements\label{chapter:package-description}}

This chapter will describe analytic element modeling with ModAEM.


\chapter{Input Formats\label{chapter:input-formats}}


\section{Input Files\label{sec:input-files}}

This section describes the files that are required as input for ModAEM.


\subsection{MODAEM.NAM\label{sec:modaem.nam}}

This file is used by ModAEM to retrieve the name of the file which contains
the input which describes the analytic element model. At this time, the file
name 'modaem.nam' is hard coded into ModAEM. modaem.nam is a flat text file
that can be created with any text editor. modaem.nam must have a single line
of text which specifies the name of the AEM input file. For example, if the
model input data are contained in a file called 'modaem.aem', then the contents
of modaem.nam would be:

\begin{lyxcode}
modaem
\end{lyxcode}
The extension '.aem' is appended to the file name found in modaem.nam.


\section{AEM input file\label{sec:AEM-input-file}}

The AEM input file is used to input the model elements and commands to ModAEM.
The AEM input file can have any name (specified in the modaem.nam file) with
the extension '.aem'. The AEM input file is a flat text file that can be created
with any text editor. Commands are entered one to a line, with carriage-control/line-feed
characters at the end of each line. The valid commands that can be entered are
as follows.


\subsection{Input commands\label{sec:input-commands}}

Input commands are entered into the AEM input file using any standard text editor.
Some commands can be (and indeed must be) nested within other commands. For
instance, commands which create elements must be nested within the 'aem' command.
It is recommended (though not required) that nested commands be indented. Commands
which start modules (i.e. 'aem') are ended with the 'end' command. Commands
are not case-sensitive - upper or lower case (or mixed case) can be used.


\subsection{Commands which are common to all input modules\label{module:common}}

The following commands are available in all ModAEM input modules.


\subsubsection{Comments\label{command:common-comment}}

Comment lines in the AEM input file start with a hash mark (\#) in the first
column. Comment lines are ignored. For example:

\begin{lyxcode}
\#~This~is~a~comment~line
\end{lyxcode}

\subsubsection{END -- Exit a module section\label{command:common-end}}

The END directive causes ModAEM to leave the module that it is currently in.
For example, when in the WL0 module (which is started with the WL0 command,
the END command signals the end of input for the element modules:

\begin{lyxcode}
\#~the~aem~section~is~used~to~define~the~problem

aem

~~\#~other~module~sections~go~here...

~~wl0~10

~~~~...~well~data~goes~in~here~...

~~~~\#~end~of~well~data

~~end

~~\#~end~of~aem~data

end

\#~processing~directives~go~here


\end{lyxcode}

\subsubsection{PCD -- Set program behavior on runtime error detections\label{command:common-pcd}}

Sets the 'proceed' flag for the error handler (useful for debugging input files).
The standard behavior of ModAEM is to terminate on any error detected during
execution. The PCD command takes a single logical argument:

\begin{lyxcode}
\#~Allow~execution~to~proceed~on~error

PCD~t

\#~Terminate~execution~on~error

PCD~f
\end{lyxcode}

\subsubsection{DBG -- enable debug code\label{command:common-dbg}}

The DBG command is used to turn code marked as 'debug' code on or off during
execution (useful for program debugging). The ModAEM code contains many assertions
which test to ensure that the ModAEM's internal data structures are 


\subsection{Top-level ModAEM commands\label{module:top-level}}

In ModAEM, each element module has its own 'Read' routine, which populates the
data structures associated with that module. The top--level ModAEM commands
discussed here are used only when ModAEM is running as a stand-alone program. 


\subsubsection{AEM -- begin defining a model problem domain\label{command:top-aem}}

The AEM command is used to enter the AEM\_Read routine, in order to populate
all elements associated with the model domain (see the AEM module discussion
in Section \ref{module:aem}).

Usage (this command takes no arguments):

\begin{lyxcode}
aem

~~...~put~model~definition~commands~here~...

end
\end{lyxcode}

\subsubsection{SOL -- solve the model\label{command:top-sol}}

After a model has been defined (using the AEM module input section), it must
be solved prior to performing any analyses. ModAEM uses an iterative solution
scheme --- at each iteration, the solution is improved based on the previous
iteration.

\begin{lyxcode}
sol~{[}number~of~iterations{]}
\end{lyxcode}
Example:

\begin{lyxcode}
aem

~~...~put~model~definition~commands~here~...

end

\#~solve~using~4~iterations

sol~4
\end{lyxcode}

\subsubsection{HEA -- report the head at a point in the model\label{command:top-hea}}

Directs ModAEM to report the hydraulic head at a specified point. Note: a solution
must be present prior to using this command. The command:

\begin{lyxcode}
hea~{[}z{]}
\end{lyxcode}
reports the head at the complex coordinate \( z=x+iy \). Note that in Fortran
free--format reads, the two parts of the complex coordinate are provided as
\( (x,y) \) pairs. Thus, the command

\begin{lyxcode}
HEA~(100.0,100.0)
\end{lyxcode}
reports the head at the coordinate \( (100,100) \).


\subsubsection{POT -- report the discharge potential\label{command:top-pot}}

Directs ModAEM to report the discharge potential at a specified point. Note:
a solution must be present prior to using this command (see the SOL command
in Section \ref{command:top-sol}). The command:

\begin{lyxcode}
pot~{[}z{]}
\end{lyxcode}
reports the head at the complex coordinate \( z=x+iy \). Note that in Fortran
free--format reads, the two parts of the complex coordinate are provided as
\( (x,y) \) pairs. Thus, the command

\begin{lyxcode}
pot~(100.0,100.0)
\end{lyxcode}
reports the head at the coordinate \( (100,100) \).


\subsubsection{GRA -- report the numerical gradient (potential) \label{command:top-gra}}

Directs ModAEM to report the numerical gradient (potential) at a specified point.
Used in program debugging; the numerical gradient should have the same value
as the total discharge (see the DIS command in \ref{command:top-dis}). Note:
a solution must be present prior to using this command (see th SOL command in
Section \ref{command:top-sol}). The command:

\begin{lyxcode}
gra~{[}z{]}~delta
\end{lyxcode}
reports the gradient at the complex coordinate \( z=x+iy \). Note that in Fortran
free--format reads, the two parts of the complex coordinate are provided as
\( (x,y) \) pairs. Thus, the command

\begin{lyxcode}
gra~(100.0,100.0)~1.0
\end{lyxcode}
reports the gradient at the coordinate \( (100,100) \).

\begin{lyxcode}

\end{lyxcode}

\subsubsection{DIS -- report the complex discharge \label{command:top-dis}}

Directs ModAEM to report the complex discharge at a specified point. Note: a
solution must be present prior to using this command (see th SOL command in
Section \ref{command:top-sol}). The command:

\begin{lyxcode}
dis~{[}z{]}
\end{lyxcode}
reports the gradient at the complex coordinate \( z=x+iy \). Note that in Fortran
free--format reads, the two parts of the complex coordinate are provided as
\( (x,y) \) pairs. Thus, the command

\begin{lyxcode}
dis~(100.0,100.0)
\end{lyxcode}
reports the discharge at the coordinate \( (100,100) \).

\begin{lyxcode}
DIS~(100.0,100.0)
\end{lyxcode}

\subsubsection{FLO -- report the total flow\label{command:top-flo}}

Directs ModAEM to report the total flow between two specified points. SOL must
be called before FLO. Note: a solution must be present prior to using this command
(see th SOL command in Section \ref{command:top-sol}). The command:

\begin{lyxcode}
flo~{[}z1{]}~{[}z2{]}~
\end{lyxcode}
reports the gradient at the complex coordinate \( z_{1}=x_{1}+iy_{1} \) and
\( z_{2}=x_{2}+iy_{2} \). Note that in Fortran free--format reads, the two
parts of the complex coordinate are provided as \( (x,y) \) pairs. Thus, the
command

\begin{lyxcode}
flo~(50.0,50.0)~(100.0,100.0)
\end{lyxcode}
reports the discharge between the coordinates \( (50,50) \) and \( (100,100) \).


\subsubsection{GRI -- Enter the grid generation module\label{command:top-gri}}

The GRI command enters the submodule GRI, which provides for the creation of
2--D SURFER\texttrademark-- or MATLAB\texttrademark--compatible grid files.
See Section \ref{module:gri} for details. Usage:

\begin{lyxcode}
GRI
\end{lyxcode}

\subsection{Module GRI -- Make a SURFER\texttrademark-- or MATLAB\texttrademark--compatible
grid\label{module:gri}}

This command instructs ModAEM to enter the grid module. Within the grid module,
grids of various results values from the model are created. The GRI command
must have a matching END command. Within the grid module, the window to be gridded
must be specified, along with the dimension of the grid. The following commands
occur within the grid module.


\subsubsection{OPT -- specifiy SURFER\texttrademark-- or MATLAB\texttrademark--compatible
grid}

This command instructs the grid module which type of grid to create. Usage:

\begin{lyxcode}
OPT~grid-type
\end{lyxcode}
\begin{description}
\item [grid-type]Choose 'surfer' for SURFER\texttrademark--compatible grids and 'matlab'
for MATLAB\texttrademark--compatible grids. SURFER\texttrademark--compatible
output files will be named with the '.GRD' extension while for MATLAB\texttrademark--compatible
output files will be named with the '.m' extension. If the OPT command is omitted,
the grid-type will default to 'surfer'.
\end{description}

\subsubsection{WIN -- Define the grid window\label{command:gri-win} }

Defines the window to be gridded. The command :

\begin{lyxcode}
win~{[}z1{]}~{[}z2{]}
\end{lyxcode}
sets the lower-left and upper-right corners of the window to be gridded to the
complex coordinates \( z_{1}=x_{1}+iy_{1} \) and \( z_{2}=x_{2}+iy_{2} \).Note
that in Fortran free--format reads, the two parts of the complex coordinate
are provided as \( (x,y) \) pairs. Thus, the command

\begin{lyxcode}
win~(-100.0,-100.0)~(100.0,100.0)
\end{lyxcode}
sets the lower-left and upper-right corners of the window for the GRI module
at the coordinates \( (-100,-100) \) and \( (100,100) \), respectively.


\subsubsection{DIM -- Specify number of grid points\label{command:gri-dim}}

Sets the number of grid points along the long axis of the window. The command

\begin{lyxcode}
dim~d
\end{lyxcode}
sets the number of grid points along the long axis to the value of d, where
d is a positive integer.


\subsubsection{HEA -- Create a grid of heads\label{command:gri-hea}}


\subsubsection{POT -- Create a grid of potentials\label{command:gri-pot}}


\subsubsection{PSI -- Create a grid of stream functions\label{command:gri-psi}}


\subsubsection{Q\_X -- Create a grid of discharges\label{command:gri-q_x}}


\subsubsection{Q\_Y -- Create a grid of discharges\label{command:gri-q_y}}

Sample grid module commands to create SURFER\texttrademark--compatible grids
of heads, potentials, stream functions, and discharges:

\begin{lyxcode}
GRI

~~~WIN~(-100.0,-100.0)~(100.0,100.0)

~~~DIM~50

~~~HEA~modaem

~~~POT~modaem

~~~PSI~modaem

~~~Q\_X~modaem

~~~Q\_Y~modaem

END
\end{lyxcode}
The parameters passed to the HEA, POT, PSI, Q\_X, and Q\_Y commands specify
the name of the output file for the grids.


\subsection{Module TR0 -- Trace\label{module:tr0}}

The TR0 command instructs ModAEM to enter the trace module, which is used to
trace 2-D streamlines. The TR0 command must have a matching END command. Within
the TR0 module, the following commands are valid:


\subsubsection{WIN -- Set the tracing window. }

Default tuning parameters are derived from the window size.


\subsubsection{TUN -- Set tuning parameters}

Sets tuning parameters for the tracing algorithm. Usage:

\begin{lyxcode}
TUN~step~prox~frac~small
\end{lyxcode}
\begin{description}
\item [step]The base step size
\item [prox]The proximity (in terms of the current step size) to boundary conditions
for reducing the step size
\item [frac]The factor for step size reductions
\item [small]Smallest allowable step size
\end{description}

\subsubsection{TIM -- Specify maximum time allowed for particle tracing}


\subsubsection{POI -- Release a single particle at the specified location}


\subsubsection{LIN -- Release particles along a line}

N particles along a line


\subsubsection{GRI -- Release a grid of particles in the sub-window}


\subsubsection{WL0 -- Release N particles in reverse from the well bore of a WL0 element.}


\subsection{Module AEM -- Input a problem definition\label{module:aem}}

The AEM module corresponds to the AEM\_DOMAIN object within ModAEM. An AEM\_DOMAIN
object contains all of the information associated with a 2--D analytic element
model. As a result, all of the AEM module commands are associated with the specification
of the model elements. The AEM module is entered with the AEM command (see Section
\ref{command:top-aem})


\subsubsection{AQU -- Enter the AQU module\label{command:aem-aqu}}

The AQU command is used to enter the AQU\_Read routine, in order set properties
associated with the model domain. A number of commands can be used within the
AQU module to describe features of the model domain (see the AQU module discussion
in Section \ref{module:aqu}).

Usage:

\begin{lyxcode}
aqu~domains~base~thickness~conductivity~porosity

~~...~put~aquifer~definition~commands~here~...

end
\end{lyxcode}
The parameters on the AQU command describe the aquifer properties. The AQU command
must occur within the AEM module. The AQU command must have a corresponding
END command. 

\begin{lyxcode}

\end{lyxcode}
Parameters for the AQU command:

\begin{description}
\item [domains]The (integer) number of domains in the aquifer
\item [base]The base elevation (real) of the aquifer
\item [thickness]The thickness (real) of the aquifer
\item [conductivity]The hydraulic conductivity (real) of the aquifer
\item [porosity]Porosity (real) of the aquifer
\end{description}
Example AQU command (with no inhomogeneities - i.e. 1 domain):

\begin{lyxcode}
AQU~1~0.0~10.0~100.0~0.25

END
\end{lyxcode}
See section \ref{module:aqu} for a description of valid commands in the AQU
module.


\subsubsection{WL0 -- Enter the WL0 (discharge-specified well) module.\label{command:aem-wl0}}

This command occurs within the AEM module. It is used to enter the WL0\_Read
routine in order to set the properties of discharge-specified well elements
in the model. The WL0 command must have a matching END command. Non-comment
lines between the WL0 command and the END command specify the parameters of
the wells in the module. Usage:

\begin{lyxcode}
WL0~wells
\end{lyxcode}
Parameters for the WL0 command:

\begin{description}
\item [wells]The (integer) number of wells
\end{description}
Sample WL0 command:

\begin{lyxcode}
WL0~10

~~~...~well~info~goes~here~...

END
\end{lyxcode}
The input format for WL0 well data is described in section \ref{module:pd0}


\subsubsection{WL1 -- Enter the WL1 (Head-specified well) module.\label{command:aem-wl1}}

This command occurs within the AEM module. It is used to enter the the WL1\_Read
routine in order to set the properties of head-specified well elements in the
model. The WL1 command must have a matching END command. Non-comment lines between
the WL1 command and the END command specify the parameters of the wells in the
module. Usage:

\begin{lyxcode}
WL1~wells
\end{lyxcode}
Parameters for the WL1 command:

\begin{description}
\item [wells]The (integer) number of wells
\end{description}
Example WL1 command:

\begin{lyxcode}
WL1~10

~~~...~Well~info~goes~here~...

END
\end{lyxcode}
The input for for WL1 well data is described in section \ref{module:wl1}


\subsubsection{LS0 -- Enter the LS0 (discharge-specified line-sink) module\label{command:aem-ls0}}

This command occurs within the AEM module. It is used to enter the LS0\_Read
routine in order to set the properties of discharge-specified line-sink elements
in the model. The LS0 command must have a matching END command. Usage:

\begin{lyxcode}
LS0~strings
\end{lyxcode}
Parameters for the LS0 command

Parameters for the LS0 command:

\begin{description}
\item [strings]The number of strings of line-sinks in the LS0 module.
\end{description}
Sample LS0 command

\begin{lyxcode}
LS0~10~~~

~~~...~String~info~goes~here~...

END
\end{lyxcode}
The input format for LS0 data is specified in section \ref{module:ls0}


\subsubsection{LS1 -- Enter the LS1 (head-specified line-sink) module\label{command:aem-ls1} }

This command occurs within the AEM module. It is used to enter the LS1\_Read
routine in order to set the properties of head-specified line-sink elements
in the model. The LS1 command must have a matching END command. Usage:

\begin{lyxcode}
LS1~strings
\end{lyxcode}
Parameters for the LS1 command:

\begin{description}
\item [strings]The number of strings of line-sinks
\end{description}
Sample LS1 command:

\begin{lyxcode}
LS1~10

~~~...~string~info~goes~here~...

END
\end{lyxcode}
The input format for LS1 data is specified in section \ref{module:hb0}.


\subsubsection{LS2 - Enter the resistance line-sink module\label{command:aem-ls2}}

This command occurs within the AEM module. It is used to enter the properties
of head-specified line-sink elements with resistance. The LS2 command must have
a matching END command. Usage:

\begin{lyxcode}
LS2~strings
\end{lyxcode}
Parameters for the LS2 command:

\begin{description}
\item [strings]The number of strings of line-sinks
\end{description}
Sample LS1 command:

\begin{lyxcode}
LS2~10

~~~...~string~info~goes~here~...

END
\end{lyxcode}
The input format for LS2 Strings is specified in section \ref{module:ls2}


\subsubsection{HB0 - Enter the HB0 (No-flow boundary) module\label{command:aem-hb0} }

This command is used to enter the HB0\_Read routine in order to set the properties
of no-flow boundaries in the model. The HB0 command must have a matching END
command. Usage:

\begin{lyxcode}
HB0~strings
\end{lyxcode}
Parameters for the HB0 command:

\begin{description}
\item [strings]The number of strings of vertices defining the no-flow boundaries
\end{description}
Sample HB0 command

\begin{lyxcode}
HB0~10~

~~~...~String~info~goes~here~...

END
\end{lyxcode}
The input format for data lines in HB0 module is specified in section \ref{module:hb0}.


\subsubsection{PD0 -- Enter the PD0 (Discharge-specified pond) module\label{command:aem-pd0}}

This command occurs within the AEM module. It is used to enter the PD0\_Read
routine in order to set the properties of discharge-specified pond elements
in the model. The PD0 command must have a matching END command. Non-comment
lines between the PD0 command and the END command specify the parameters of
the wells in the module. Usage:

\begin{lyxcode}
PD0~ponds
\end{lyxcode}
Parameters for the PD0 command:

\begin{description}
\item [ponds]The (integer) number of ponds
\end{description}
Sample PD0 command:

\begin{lyxcode}
PD0~10

~~~...~pond~info~goes~here~...

END
\end{lyxcode}
The input format for PD0 pond data is described in section\ref{module:pd0}


\subsubsection{AS0 -- Enter the AS0 (discharge-specified area sink) module\label{command:aem-as0} }

This command occurs within the AEM module. It is used to enter the AS0\_Read
routine in order to set the properties of discharge-specified area sink elements
in the model. The AS0 command must have a matching END command. Non-comment
lines between the AS0 command and the END command specify the vertices which
define the boundaries of the discharge-specified area sink. Usage:

\begin{lyxcode}
AS0~Vertices~Top-Recharge~Bottom-Recharge
\end{lyxcode}
Parameters for the AS0 command:

\begin{description}
\item [Vertices]The (integer) number of vertices which delineate the boundaries of
the area sink
\item [Top-Recharge]The (real) recharge rate through the top of the area sink.
\item [Bottom-Recharge]The (real) recharge rate through the bottom of the area sink.
\end{description}
Sample AS0 command:

\begin{lyxcode}
AS0~10~1.0~1.0

~~~(10,10)

~~~(20,20)

~~~...~

END
\end{lyxcode}
The data lines within the AS0 module take the form:

\begin{lyxcode}
{[}z{]}
\end{lyxcode}
where the location of the vertex is given by \( z=x+iy \) with given discharge
(real), radius (real), and id (integer).Note that in Fortran free--format reads,
the two parts of the complex coordinate are provided as \( (x,y) \) pairs.
Thus, the command

\begin{lyxcode}
(10,10)
\end{lyxcode}
defines a vertex of the boundary of the discharge-specified area sink at \( (10,10) \).


\subsection{AQU Module\label{module:aqu}}


\subsubsection{REF -- Reference point command\label{command:aqu-ref}}

The REF command occurs within the AQU module. The REF command is optional. It
gives the head at a reference point and the (optional) uniform flow for the
aquifer. Usage:

\begin{lyxcode}
REF~{[}z{]}~head~\{{[}Q{]}\}
\end{lyxcode}
Parameters for the REF command:

\begin{description}
\item [{[}z{]}]the location, \( z=x+iy \) of the reference point. Note that in Fortran
free--format reads, the two parts of the complex coordinate are provided as
\( (x,y) \) pairs. 
\item [head]The (real) head at the reference point.
\item [{[}Q{]}(optional)]The (complex) uniform flow, \( Q=Q_{x}+iQ_{y} \) at the
reference point. Note that in Fortran free--format reads, the two parts of the
complex coordinate are provided as \( (x,y) \) pairs. 
\end{description}
Sample REF command:

\begin{lyxcode}
REF~(0,0)~100.0
\end{lyxcode}
Establishes a reference point at coordinate \( (0,0) \) with head \( 100 \),
and gives no uniform flow.

\emph{Vic - you'd better explain the uniform flow argument.}


\subsubsection{BDY - Boundary command\label{command:aqu-bdy}}

The BDY command is used to delineate the elements at the boundary of a finite
aquifer. It must occur within the AQU module. The BDY command is followed by
a string of data records and ends when another command is encountered. Each
data line consists of two end-points, a reference head (real), a flux (real)
and a logical flag . Sample BDY command:

\begin{lyxcode}
AQU~...

~~~REF~...

~~~BDY

~~~~~~(10,10)~(20,20)~100.0

~~~~~~(20,20)~(30,30)~99.5

~~~~~~~...

END
\end{lyxcode}
\emph{Vic - here's another one that could use a better explanation than I know
how to give.}


\subsubsection{PRM -- Perimeter command\label{command:aqu-prm} }

The PRM command is used to delineate the perimeter of a finite aquifer. It occurs
within the AQU module. An error results if the perimeter is specified and the
user requests any flow conditions outside the perimeter. Usage:

\begin{lyxcode}
PRM~default-val
\end{lyxcode}
Parameters for the PRM command:

\begin{description}
\item [default-val]The default value to be used if the user requests a value (head,
potential, recharge, velocity, discharge) at a point outside the perimeter.
Use an easily recognizable number (e.g. -9999).
\end{description}
Sample PRM command:

\begin{lyxcode}
AEM~...

~~~REF~...

~~~PRM~-9999

~~~~~~(1000,1000)

~~~~~~(1000,-1000)

~~~~~~(-1000,~-1000)

~~~~~~(-1000,~1000)

END
\end{lyxcode}

\subsubsection{IN0 -- Enter the IN0 (inhomogeneity) module\label{command:aqu-in0}}

Used to enter the inhomogeneity module. The IN0 command occurs within the AQU
module and must have a corresponding END command. This command takes no parameters.


\subsection{IN0 Module\label{module:in0}}

Inhomogeneities in the model domain are delineated in the IN0 module.


\subsubsection{DOM -- Define inhomogeneity.}

The DOM command is used to define an inhomogeneity in the aquifer. It must occur
within the IN0 module. There should be one DOM command for each inhomogeneity
specified in the AQU command. The DOM command is allow only within the IN0 module.
Usage:

\begin{lyxcode}
DOM~Vertices~Base~Thickness~Conductivity~Porosity
\end{lyxcode}
Parameters for the DOM command:

\begin{description}
\item [Domains]The (integer) number of vertices which delineate the boundaries of
the inhomogeneity
\item [Base]The base elevation (real) of the aquifer
\item [Thickness]The thickness (real) of the aquifer
\item [Conductivity]The hydraulic conductivity (real) of the aquifer
\item [Porosity]Porosity (real) of the aquifer
\end{description}
Sample DOM command:

\begin{lyxcode}
AQU~2~0.0~10.0~100.0~0.25

~~~IN0

~~~~~~DOM~10~0.0~10.0~50.0~0.25

~~~~~~~~~(10,10)

~~~~~~~~~(20,20)

~~~~~~~~~...~

~~~END

END
\end{lyxcode}

\subsection{WL0 Module\label{module:wl0}}

The WL0 module is used to add discharge-specified wells to the analytic element
model. See section \ref{command:aem-wl0}for a description of the WL0 command
used to enter the WL0 module. Within the WL0 module, only data lines can be
entered. Data lines take the form:

\begin{lyxcode}
{[}z{]}~discharge~radius~id
\end{lyxcode}
defines a discharge-specified well at location \( z=x+iy \) with given discharge
(real), radius (real), and id (integer).Note that in Fortran free--format reads,
the two parts of the complex coordinate are provided as \( (x,y) \) pairs.
Thus, the command

\begin{lyxcode}
(50,50)~100.0~0.1~1~
\end{lyxcode}
defines well \( 1 \) at coordinate \( (50,50) \) with discharge \( 100.0 \)
and radius \( 0.1 \).


\subsection{WL1 Module\label{module:wl1}}

The WL1 module is used to add head-specified wells to the analytic element model.
See section \ref{command:aem-wl1} for a description of the WL1 command used
to enter the WL1 module. Within the WL1 module, only data line can be entered.
Data lines take the form:

\begin{lyxcode}
{[}z1{]}~radius~{[}z2{]}~head~id
\end{lyxcode}
defines a well at location \( z_{1}=x_{1}+iy_{1} \) with given radius (real)
and with head (real) specified at location \( z_{2}=x_{2}+iy_{2} \)and id (integer).
Note that in Fortran free--format reads, the two parts of the complex coordinate
are provided as \( (x,y) \) pairs. Thus, the line

\begin{lyxcode}
(50.0,50.0)~0.1~(100.0,100.0)~100.0~1
\end{lyxcode}
defines well \( 1 \) at coordinate \( (50,50) \) with radius \( 0.1 \) and
with the head specified at 100 at coordinate \( (100,100) \).


\subsection{LS0 Module\label{module:ls0}}

The LS0 module is used to add discharge-specified line sink elements within
the AEM module. See section \ref{command:aem-ls0} for a description of the
LS0 command used to enter the LS0 module. Within the LS0 module, the only command
allowed is the STR command, which is used to define a string of line sinks. 


\subsubsection{STR Command -- string of discharge-specified line sinks.}

The STR command within the LS0 module is used to add strings of discharge-specified
line sinks to the model domain. Usage:

\begin{lyxcode}
STR~vertices~id
\end{lyxcode}
parameters for the STR command:

\begin{description}
\item [vertices]the number of vertices (integer) for the string
\item [id]the string id (integer)
\end{description}
Sample STR command:

\begin{lyxcode}
LS0~10

~~~STR~2~1

~~~~~~(10,10)~100

~~~~~~(20,20)~95

~~~STR~10~2

~~~~~~...

~~~STR...~~~

END


\end{lyxcode}
The data records for which occur after each STR command specify the vertices
for the line sinks within that string. The data record takes the format:

\begin{lyxcode}
{[}z{]}~discharge
\end{lyxcode}
defines a one vertex of a string of discharge-specified line sinks at location
\( z=x+iy \) with given discharge (real). Note that in Fortran free--format
reads, the two parts of the complex coordinate are provided as \( (x,y) \)
pairs.


\subsection{LS1 Module\label{module:ls1}}

The LS1 module is used to add head-specified line sink elements within the AEM
module. See section \ref{command:aem-ls1}for a description of the LS1 command
used to enter the LS1 module. Within the LS1 module, the only command allowed
is the STR command, which is used to define a string of line sinks. 


\subsubsection{STR Command -- string of head-specified line sinks.}

The STR command within the LS1 module is used to add strings of discharge-specified
line sinks to the model domain. Usage:

\begin{lyxcode}
STR~vertices~id
\end{lyxcode}
parameters for the STR command:

\begin{description}
\item [vertices]the number of vertices (integer) for the string
\item [id]the string id (integer)
\end{description}
Sample STR command:

\begin{lyxcode}
LS1~10

~~~STR~2~1

~~~~~~(10,10)~100

~~~~~~(20,20)~95

~~~STR~10~2

~~~~~~...

~~~STR~...~~~

END


\end{lyxcode}
The data records for which occur after each STR command specify the vertices
for the line sinks within that string. The data record takes the format:

\begin{lyxcode}
{[}z{]}~head
\end{lyxcode}
defines one vertex of a string of head-specified line sink at location \( z=x+iy \)
with given head (real). Note that in Fortran free--format reads, the two parts
of the complex coordinate are provided as \( (x,y) \) pairs.


\subsection{LS2 Module\label{module:ls2}}

The LS2 module is used to add head-specified line sink elements with resistance
within the AEM module. See section \ref{command:aem-ls2}for a description of
the LS2 command used to enter the LS2 module. Within the LS2 module, the only
command allowed is the STR command, which is used to define a string of line
sinks. 


\subsubsection{STR Command -- string of head-specified line sinks.}

The STR command within the LS1 module is used to add strings of discharge-specified
line sinks to the model domain. Usage:

\begin{lyxcode}
STR~vertices~c~w~d~route-id~drain-en~route-en~id
\end{lyxcode}
Parameters for the STR command:

\begin{description}
\item [vertices]The maximum number of vertices along the modeled stream reach
\item [c]The 'resistance' of the line-sink, in units of time (e.g. days). This is
the reciprocal of the MODFLOW 'leakance', and is computed as \( \frac{Thickness}{K} \)where
\( Thickness \) is the thickness of the resistance layer and \( K \) is the
vertical hydraulic conductivity? of the resistance layer)
\item [w]The width of the stream
\item [d]The 'depth', defined as the distance from the water level in the stream (the
specified heads at the vertices) to the bottom of the resistance layer. This
is used to determine whether the line-sink is 'percolating', in a manner similar
to the MODFLOW RIV package.
\item [route-id]The ID number of the stream reach below this one in a stream network.
Set route-id = 1 if there is no downstream route. 
\item [drain-en]Flag for 'drain' elements. .true. for drains, .false. if not. if the
drain-en flag is set (\emph{Vic - does this mean true}?), the line-sink cannot
lose water (as in MODFLOW DRN package).
\item [route-en]Flag for enabling routine decision-making. This will allow line-sinks
to be 'turned off' if there is no net flow in the stream reach, according to
the routing package. This behavior is analogous to the MODFLOW STR package.
\item [id]The ID number for the stream reach. Note that this is the tool for linking
in the routing package.
\end{description}

\subsection{HB0 Module\label{module:hb0}}

The HB0 module is used to no-flow boundary elements within the AEM module. See
section \ref{command:aem-hb0}for a description of the HB0 command used to enter
the HB0 module. Within the HB0 module, the only command allowed is the STR command,
which is used to define a string of vertices defining a no-flow boundary. 


\subsubsection{STR Command -- string of no-flow boundary vertices.}

The STR command within the HB0 module is used to add strings of vertices which
define no-flow boundaries to the model domain. Usage:

\begin{lyxcode}
STR~vertices~id
\end{lyxcode}
parameters for the STR command:

\begin{description}
\item [vertices]the number of vertices (integer) for the string
\item [id]the string id (integer)
\end{description}
Sample STR command:

\begin{lyxcode}
HB0~10

~~~STR~2~1

~~~~~~(10,10)

~~~~~~(20,20)

~~~STR~10~2

~~~~~~...

~~~STR~...~~~decison

END
\end{lyxcode}
The data records for which occur after each STR command specify the vertices
for the no-flow boundary defined by that string. The data record takes the format:

\begin{lyxcode}
{[}z{]}
\end{lyxcode}
defines one vertex at location \( z=x+iy \). Note that in Fortran free--format
reads, the two parts of the complex coordinate are provided as \( (x,y) \)
pairs.


\subsection{PD0 Module\label{module:pd0}}

The PD0 module is used to add discharge-specified ponds to the analytic element
model. See section \ref{command:aem-pd0}for a description of the PD0 command
used to enter the PD0 module. Within the PD0 module, only data lines can be
entered. Data lines take the form:

\begin{lyxcode}
{[}z{]}~discharge~radius~id
\end{lyxcode}
defines a discharge-specified pond with center at location \( z=x+iy \) with
given discharge (real), radius (real), and id (integer). Note that in Fortran
free--format reads, the two parts of the complex coordinate are provided as
\( (x,y) \) pairs. Thus, the command

\begin{lyxcode}
(50,50)~100.0~100.0~1~
\end{lyxcode}
defines pond \( 1 \) at coordinate \( (50,50) \) with discharge \( 100.0 \)
and radius \( 100.0 \).


\chapter{Output Formats}

This chapter will describe ModAEM output files.


\chapter{Validation}

This chapter will describe validation of ModAEM results.
\end{document}
